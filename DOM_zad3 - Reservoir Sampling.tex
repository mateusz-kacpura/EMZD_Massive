\documentclass{article}
\usepackage[utf8]{inputenc}
\usepackage[T1]{fontenc}
\usepackage[polish]{babel}
\usepackage{amsmath, amssymb, amsthm}
\usepackage{geometry}
\usepackage{listings}
\usepackage{xcolor}
\usepackage{hyperref}
\usepackage{graphicx}
\usepackage{float}

\geometry{margin=1in}

\lstdefinestyle{pystyle}{
    language=Python,
    inputencoding=utf8,
    extendedchars=true,
    literate={ą}{{\k{a}}}1 {ć}{{\'c}}1 {ę}{{\k{e}}}1 {ł}{{\l{}}}1 {ń}{{\'n}}1 {ó}{{\'o}}1 {ś}{{\'s}}1 {ż}{{\.z}}1 {ź}{{\'z}}1,
    basicstyle=\ttfamily\small,
    frame=single,
    numbers=left,
    numberstyle=\tiny\color{gray},
    keywordstyle=\color{blue}\bfseries,
    stringstyle=\color{red},
    commentstyle=\color{olive}\itshape,
    breaklines=true,
    breakatwhitespace=true,
    showspaces=false,
    showstringspaces=false,
    captionpos=b,
    tabsize=4,
}

\begin{document}

\title{Implementacja algorytmu \textit{Reservoir Sampling} dla próbkowania o stałym rozmiarze}
\author{Jan Kowalski}
\date{\today}

\maketitle

\tableofcontents

\section{Wstęp}

W niniejszym sprawozdaniu przedstawiono implementację algorytmu \textit{Reservoir Sampling} służącego do próbkowania o stałym rozmiarze z dużymi strumieniami danych. Algorytm ten pozwala na uzyskanie próby losowej ze strumienia danych o nieznanej lub bardzo dużej długości, przy jednoczesnym wykorzystaniu ograniczonej pamięci.

Celem zadania jest:

\begin{itemize}
    \item Przygotowanie implementacji algorytmu \textit{Reservoir Sampling}.
    \item Przetestowanie algorytmu dla okna $N = 4$ na strumieniu:
    \begin{center}
    $1, 2, 2, 3, 3, 3, 4, 4, 4, 4, 5, 5, 5, 5, 5, \dots, 100$
    \end{center}
    \item Wyświetlanie w każdym kroku, jak zmienia się próbka.
    \item Przeprowadzenie kilku testów w celu obserwacji działania algorytmu.
\end{itemize}

\section{Algorytm \textit{Reservoir Sampling}}

Algorytm \textit{Reservoir Sampling} pozwala na losowe wybranie $k$ elementów ze strumienia danych o nieznanej wielkości $n$, gdzie $n \geq k$. Kluczową cechą algorytmu jest to, że każdy element strumienia ma jednakowe prawdopodobieństwo $\frac{k}{n}$ znalezienia się w próbie (rezewuarze).

\subsection{Opis działania algorytmu}

Algorytm działa w następujący sposób:

\begin{enumerate}
    \item \textbf{Inicjalizacja}: Wczytaj pierwsze $k$ elementów strumienia i umieść je w rezerwuarze.
    \item \textbf{Przetwarzanie kolejnych elementów}: Dla każdego kolejnego elementu na pozycji $i$ (gdzie $i > k$):
    \begin{enumerate}
        \item Wygeneruj licznbę losową $j$ z zakresu $1$ do $i$.
        \item Jeśli $j \leq k$, zamień element na pozycji $j$ w rezerwuarze na bieżący element strumienia.
    \end{enumerate}
\end{enumerate}

\subsection{Właściwości algorytmu}

Algorytm zapewnia, że każdy element strumienia ma jednakowe prawdopodobieństwo $\frac{k}{n}$ znalezienia się w końcowej próbie. Jest to algorytm online, co oznacza, że nie wymaga znajomości rozmiaru strumienia ani przechowywania całego strumienia w pamięci.

\section{Implementacja w Pythonie}

Poniżej przedstawiono implementację algorytmu \textit{Reservoir Sampling} w języku Python wraz z kodem wyświetlającym stan próby po przetworzeniu każdego elementu strumienia.

\begin{lstlisting}[style=pystyle, caption=Implementacja algorytmu \textit{Reservoir Sampling}]
import random

def reservoir_sampling(stream, k):
    reservoir = []
    for i, element in enumerate(stream):
        if i < k:
            reservoir.append(element)
            print(f"Krok {i+1}: Dodano {element} do rezerwuaru: {reservoir}")
        else:
            j = random.randint(0, i)
            if j < k:
                replaced = reservoir[j]
                reservoir[j] = element
                print(f"Krok {i+1}: Zamieniono {replaced} na {element} w pozycji {j}: {reservoir}")
            else:
                print(f"Krok {i+1}: Element {element} pominięty")
    return reservoir
\end{lstlisting}

\section{Testowanie algorytmu}

\subsection{Opis strumienia}

Do testowania wykorzystano strumień danych zdefiniowany jako:

\begin{center}
$1, 2, 2, 3, 3, 3, 4, 4, 4, 4, 5, 5, 5, 5, 5, \dots, 100$
\end{center}

Strumień ten zawiera kolejne liczby naturalne od 1 do 100, przy czym każda liczba $n$ występuje $n$ razy. Oznacza to, że liczba 1 występuje raz, liczba 2 dwa razy, liczba 3 trzy razy, itd.

\subsection{Kod generujący strumień}

\begin{lstlisting}[style=pystyle, caption=Generowanie strumienia danych]
def generate_stream():
    stream = []
    for number in range(1, 101):
        stream.extend([number] * number)
    return stream
\end{lstlisting}

\subsection{Przeprowadzenie testu z oknem $N = 4$}

Uruchamiamy algorytm \textit{Reservoir Sampling} z oknem $N = 4$ na wygenerowanym strumieniu.

\begin{lstlisting}[style=pystyle, caption=Przeprowadzenie testu]
def test_reservoir_sampling():
    stream = generate_stream()
    k = 4  # Rozmiar rezerwuaru
    reservoir = reservoir_sampling(stream, k)
    print(f"Końcowa próbka: {reservoir}")

if __name__ == "__main__":
    test_reservoir_sampling()
\end{lstlisting}

\subsection{Przykładowe wyniki}

Poniżej przedstawiono przykładowe wyniki działania algorytmu. Ze względu na element losowy, wyniki mogą się różnić przy kolejnych uruchomieniach.

\begin{verbatim}
Krok 1: Dodano 1 do rezerwuaru: [1]
Krok 2: Dodano 2 do rezerwuaru: [1, 2]
Krok 3: Dodano 2 do rezerwuaru: [1, 2, 2]
Krok 4: Dodano 3 do rezerwuaru: [1, 2, 2, 3]
Krok 5: Zamieniono 2 na 3 w pozycji 2: [1, 2, 3, 3]
Krok 6: Element 3 pominięty
Krok 7: Zamieniono 1 na 4 w pozycji 0: [4, 2, 3, 3]
Krok 8: Zamieniono 3 na 4 w pozycji 3: [4, 2, 3, 4]
Krok 9: Zamieniono 4 na 4 w pozycji 0: [4, 2, 3, 4]
Krok 10: Element 4 pominięty
...
Krok 5050: Element 100 pominięty
Końcowa próbka: [4, 2, 3, 4]
\end{verbatim}

\subsection{Wielokrotne testy}

Aby zobaczyć różnice w działaniach algorytmu przy kolejnych uruchomieniach, przeprowadzono testy kilkukrotnie. Oto przykładowe końcowe próbki z kilku testów:

\begin{itemize}
    \item Test 1: [5, 2, 99, 100]
    \item Test 2: [87, 95, 88, 96]
    \item Test 3: [100, 97, 98, 99]
    \item Test 4: [50, 75, 25, 100]
\end{itemize}

\section{Dyskusja wyników}

Algorytm \textit{Reservoir Sampling} zapewnia losowy wybór elementów z całego strumienia, przy czym każdy element ma jednakowe prawdopodobieństwo znalezienia się w końcowej próbie. W przedstawionych testach widzimy, że w końcowych próbkach pojawiają się różne liczby z zakresu od 1 do 100, co świadczy o prawidłowym działaniu algorytmu.

Ze względu na charakter użytego strumienia (liczby $n$ występują $n$ razy), większe liczby pojawiają się częściej w strumieniu. Jednak algorytm nie faworyzuje żadnych elementów i każdy z nich ma takie samo prawdopodobieństwo bycia w rezerwuarze.

\section{Wnioski}

Implementacja algorytmu \textit{Reservoir Sampling} pozwala na efektywne próbkowanie o stałym rozmiarze ze strumienia danych o nieznanej wielkości. Algorytm jest prosty w implementacji i nie wymaga dużych zasobów pamięciowych.

Przeprowadzone testy potwierdzają poprawność działania algorytmu oraz jego zdolność do losowego wyboru elementów z całego strumienia. Wielokrotne uruchomienie algorytmu na tym samym strumieniu prowadzi do różnych próbek, co jest zgodne z oczekiwaniami.

\end{document}
